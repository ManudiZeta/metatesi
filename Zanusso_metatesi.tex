%----------------------------------------------------------------------------------------
%	PACKAGES AND THEMES
%----------------------------------------------------------------------------------------
\documentclass[aspectratio=169,xcolor=dvipsnames, t]{beamer}
\usepackage{fontspec} % Allows using custom font. MUST be before loading the theme!
\usetheme{SimplePlusAIC}
\usepackage{hyperref}
\usepackage{mathtools}
\usepackage{enumerate}
\usepackage{graphicx} % Allows including images
\usepackage{booktabs} % Allows the use of \toprule, \midrule and  \bottomrule in tables
\usepackage{svg} %allows using svg figures
\usepackage{tikz}
\usepackage{makecell}
% ADD YOUR PACKAGES BELOW
\usepackage{wrapfig}
\usepackage[export]{adjustbox}

\usepackage[
  backend=biber,
  style=numeric,
  sorting=none
]{biblatex}

\addbibresource{references.bib}


\newcommand{\backupbegin}{
   \newcounter{finalframe}
   \setcounter{finalframe}{\value{framenumber}}
}
\newcommand{\backupend}{
   \setcounter{framenumber}{\value{finalframe}}
}

\newcommand{\inlineimg}[1]{%
  \raisebox{-0.2em}{\includegraphics[height=1em]{#1}}%
}
%----------------------------------------------------------------------------------------
%	TITLE PAGE CONFIGURATION
%----------------------------------------------------------------------------------------

\title[short title]{Characterization of Hadronic Showers in the Belle II Electromagnetic Calorimeter}  

\author{Emanuele Zanusso}
\institute[Characterization of Hadronic Showers in the Belle II Electromagnetic Calorimeter - Emanuele Zanusso]{Dipartimento di Fisica
\newline
Università degli Studi di Torino
}
% Your institution as it will appear on the bottom of every slide, maybe shorthand to save space


\date{January 14, 2026} % Date, can be changed to a custom date
%----------------------------------------------------------------------------------------
%	PRESENTATION SLIDES
%----------------------------------------------------------------------------------------

\begin{document}

\maketitlepage

\begin{frame}[t,noframenumbering]{Outline}
    \tableofcontents
\end{frame}

\makesection{Anti-neutrons in physics experiments}

%------------------------------------------------
\begin{frame}{Anti-neutron in HEP experiments}
The $\bar{n}$ plays a key role in several physics measurements, such as: 

    \begin{itemize}    
    \item The neutron e.m. form factor studies in $e^+ + e^- \rightarrow n + \bar{n}$ process
        	\begin{figure}[p]
     	   \includegraphics[scale=0.18, angle=0]{images/feynman-eN.png}
	\end{figure}
  
    \item Some decay channels studied at B-factories which involve $\bar{n}$
    
    \begin{enumerate}
    	\item The hyperons decay channel:
	\begin{center}
		$\bar{\Lambda}^0 \rightarrow \pi^0 + \bar{n}$, \hspace{1cm} $\bar{\Sigma}^- \rightarrow \pi^- + \bar{n}$, \hspace{1cm} $\bar{\Lambda}_c \rightarrow  K_s^0 + \pi^0 +\bar{n} $
	\end{center}
	\end{enumerate}
	
\item Discrimination between other neutral particles ($\gamma$) and $\bar{n}$

    \end{itemize}
\end{frame}

%------------------------------------------------
\begin{frame}{Anti-neutrons in astrophysics}
The $\bar{n}$ also plays a key role in several astrophysics measurements, such as:
\vspace{0.2cm}
    \begin{itemize}    
    \item Studying $\bar{n}$ - anti-hyperon potential to improve the understanding of the equation of state of the neutron stars 
    \vspace{0.1cm}
    \item Investigating dark matter through anti-deuterons ($\bar{D}$) in cosmic rays, produced by dark matter annihilation or decay 
    \vspace{0.3cm}

    $ A_{d.m.} \ + \ B_{d.m.} \ \rightarrow hadrons$ ($n$, $\bar{n}$, $p$, $\bar{p}$ etc...  )
    \\
     $ X_{d.m.} \ \rightarrow$ hadrons ($n$, $\bar{n}$, $p$, $\bar{p}$ etc...  )

    \vspace{0.2cm}
    

   $\bar{D}$ is mainly produced through a coalescence mechanism $\bar{n} + \bar{p} \rightarrow \bar{D}$, 
    \\
    where $\bar{p}$ and $\bar{n}$ are nearby in the phase-space

    \end{itemize}
\end{frame}

%------------------------------------------------
\begin{frame}{The Belle II experiment}
SuperKEKB is an asymmetric $e^+ \ e^-$ collider (Tsukuba, Japan)
\begin{columns}[T] % T = top alignment
    % Colonna sinistra
    \begin{column}{0.53\textwidth}
        \begin{itemize}
            \item 7 GeV electron beam (HER)
            \item 4 GeV positron beam (LER)
            \item Peak Luminosity $\sim 5.1 \times 10^{34} cm^{-2} s^{-1}$ \inlineimg{images/logos/coppa-del-mondo} 
            \item Design Luminosity $\sim 8 \times 10^{35} cm^{-2} s^{-1}$  \\ $\rightarrow$ x40 the Belle's one 
        \end{itemize}
    \end{column}
    
    % Colonna destra
    \begin{column}{0.60\textwidth}
        \includegraphics[scale=0.12, angle=0]{images/superkekb.png}
        \includegraphics[scale=0.24, angle=0]{images/int_lumi.png}
    \end{column}
\end{columns}

\vspace{0.5cm} % Spazio tra le due sezioni

% Parte inferiore con un’unica colonna
It operates mainly around $\Upsilon(4S)$ resonance ($\sim10.58$ GeV):
\begin{itemize}
    \item This resonance decays almost exclusively into entangled couples of $B \ \bar{B} \rightarrow$ B-factory
    \item Several goals: flavour physics, BSM physics, heavy hadrons spectroscopy etc...
\end{itemize}

\end{frame}

%----------------------------------------------------------------------------------------
\begin{frame}{The Belle II experiment}

\includegraphics[width=1\textwidth]{images/belleii_dets}
    
\end{frame}

%----------------------------------------------------------------------------------------

\begin{frame}{The Electromagnetic CaLorimeter}
The ECL plays a central role in this thesis
	\begin{columns}
	\begin{column}{0.45\textwidth}
	\begin{itemize}
		\item Array of \textbf{CsI(Tl)} crystals (8376 $6x6x30 cm^3$ crystals in total)
	       	\item	It covers barrel and end-cap regions ($12°\le\theta\le155°$)
		\item Energy resolution of 4\% @100 MeV and 1.6\% @8 GeV
	\end{itemize}
	\end{column}
	\begin{column}{0.65\textwidth}
	\begin{figure}[p]
\includegraphics[width = 1 \textwidth]{images/belleii_scheme}
	\end{figure}
	\end{column}
	
	\end{columns}
    
\end{frame}
%----------------------------------------------------------------------------------------
\begin{frame}{Anti-neutron interactions in physics}
The $\bar{n}$ interacts primarily via strong nuclear force, producing hadronic showers
\\
It can annihilate with nucleons in the ECL, producing light mesons (mainly pions) 
\vspace{0.2cm}

    \begin{itemize}    
	\item $\pi^0$ decays into $\gamma\gamma$, producing electromagnetic showers that are fully contained in the ECL
	\item $\pi^\pm$ undergo hadronic interactions, which are not fully contained in the ECL $\rightarrow$ both the forward (KLM) and backward (TOP) directions are involved
	\vspace{0.3cm}
    	\begin{figure}[p]
     	   \includegraphics[scale=0.16, angle=0]{images/Ann-nbar.png}
	   \hspace{1.5cm}
	    \includegraphics[scale=0.23, angle=0]{images/shanette_nbar_ECLTOP.jpg}
	\end{figure}
	 
    \end{itemize}
\end{frame}
%------------------------------------------------
\begin{frame}{Electromagnetic and hadronic showers}
Different processes occur for electromagnetic and hadronic showers:
\vspace{0.1cm}
	\begin{enumerate}
		\item  Bremsstrahlung and pair production process ($e^+,e^-,\gamma$) and $\pi^0 \rightarrow \gamma \gamma$
		\item  Strong interactions of hadrons with the material ($p,n,pions,kaons...$)
	\end{enumerate}

\begin{columns}
	\begin{column}{0.45\textwidth}
    		\begin{itemize}    
			\item About the 95\% of the hadronic shower is contained within a cylinder of radius $\lambda_{had}$ ($\sim$ 44.12 cm in CsI(Tl))
			\item About the 90\% of the e.m. shower is contained within a cylinder of radius $R_M$ ($\sim$ 3.6 cm in CsI(Tl))
	    	\end{itemize}
	\end{column}
	
	\begin{column}{0.60\textwidth}
		\begin{figure}[p]
		   \includegraphics[scale=0.45, angle=0]{images/gamma_vs_hadronic.png}
		\end{figure}
	\end{column}
\end{columns}
\end{frame}

%------------------------------------------------
\begin{frame}{The MANTRA project (PRIN2022)}
 \textbf{M}easuring \textbf{A}nti-\textbf{N}eutron: \textbf{T}agging and \textbf{R}econstruction \textbf{A}lgorithm
 
 \vspace{0.2cm}
    \begin{itemize}    
    \item A general method to measure the $E_{\bar{n}}$ up to 10 GeV, by combining information from:
    
    	\begin{enumerate}
    		\item A  detector with high time resolution (TOP)
		\item An electromagnetic calorimeter (\textbf{ECL})
		\item A muon system (KLM)
    	\end{enumerate}
	 \vspace{0.1cm}
     \item These features are common in modern general-purpose collider experiments such as \textbf{Belle II} and BESIII, which do not have a dedicated calorimeter 
     \item For MANTRA project, only signals from ECL and TOP are taken into account. In this thesis only ECL signals are studied
    \end{itemize}
    
    
    \vspace{0.4cm}
    {\large
		\textcolor{red}{ \textbf{Are $\bar{n}$ hadronic showers correctly simulated in the Belle II software?}}
    }
\end{frame}

%------------------------------------------------
\begin{frame}{The MANTRA project} 
Anti-neutrons cannot be reconstructed by sub-detectors.
\\
The measurement of the energy is a two-step process:
\vspace{0.1cm}
    	\begin{enumerate}
    		\item $\bar{n}$ identification via its induced ECL clusters (study of the shower shape)
		\vspace{0.1cm}
		\item Combine the signals from TOP and ECL to reconstruct the $\bar{n}$ energy, in cases of backscattering or pre-showering
		\vspace{0.2cm}
		\begin{itemize}
			\item If only $\pi^0$ are produced ($\sim 5 \%$), the energy is all contained in the calorimeter, the shower is fully reconstructed
			\vspace{0.2cm}
		        	\item	Otherwise ($\sim 95\%$): the products may escape the crystals \\ $\rightarrow$ the goal is to complement the calorimeter information with that from the adjacent detectors
		\end{itemize}
    	\end{enumerate}
    
\end{frame}

%------------------------------------------------
\begin{frame}{Preliminary concept}
Several channels can be selected to look at $\bar{n}$ annihilations, such as:
	\begin{itemize}
		\item $e^+ + e^-  \rightarrow p + \bar{n} + \pi^- + (\gamma_{ISR})$ (Mine)
		\item $\bar{\Lambda}_c \rightarrow  K_s^0 + \pi^0 +\bar{n} $ 
		\item $\Lambda (\rightarrow \ p + \pi^- ) + \bar{\Lambda} (\rightarrow \bar{n} + \pi^0)$ 
	\end{itemize}
	
	\vspace{0.2cm}
Several variables can be used to validate the showers shape for $\bar{n}$ identification, such as:
\vspace{0.1cm}
    	\begin{itemize}    
	{ \small
		\item \textbf{Zernike Moments} (backup)
		\item \textbf{Lateral momentum} defined as:  $C_{LM} = \frac{ \sum\limits_{i=2}^n \omega_i E_i r_i^2} { \omega_0 E_0 r_0^2 + \omega_1 E_1 r_1^2+ \sum\limits_{i=2}^n \omega_i E_i r_i^2}$    
 		\item \textbf{Second moment} defined as:  $ C_{SM} = \frac{ \sum\limits_{i=0}^n \omega_i E_i r_i^2} { \sum\limits_{i=0}^n \omega_i E_i}$  
	}
    	\end{itemize}
	
	
\end{frame}

%------------------------------------------------------------------------------------------------
\makesection{Study via signal Monte Carlo sample}
%------------------------------------------------------------------------------------------------

\begin{frame}{Analysis outline}
\begin{enumerate}
	\item Study of the selected signal channel $e^+ + e^- \rightarrow p + \bar{n} + \pi^- + (\gamma_{ISR})$
    		\begin{enumerate}[(a)]
    			\item Recoil identification from the system $p + \pi^-$ (with and without ISR) 
			\item Study of the kinematic recoil variables (momentum, angles, energy, etc...)
			\item Study of the effect of 1C kinematic fit over the recoil mass
			\item Study of ECL shower shape variables
    		\end{enumerate}
		
	\vspace{0.2cm}
	\item Study of MC cocktail sample:

	\begin{enumerate}[(a)]
			\item Recoil identification from the system $p + \pi^-$ (with and without ISR) 
			\item Study of the kinematic recoil variables (momentum, angles, energy, etc...)
	\end{enumerate}
	
	\vspace{0.2cm}
	\item Study of real data sample:
	\begin{enumerate}[(a)]
			\item Recoil identification from the system $p + \pi^-$ (with and without ISR) 
			\item Constraint with 1C kinematic fit over the recoil mass
			\item Examine Data/MC agreement in ECL cluster shapes from $\bar{n}$ channel

	\end{enumerate}
			
	
\end{enumerate}
\end{frame}

%------------------------------------------------------------------------------------------------

\begin{frame}{Analysis outline (1)}
\begin{itemize}

\item The analyzed channel is:

  \begin{center}
    $e^+ + e^-  \rightarrow p + \bar{n} + \pi^- + (\gamma_{ISR})$ 
  \end{center}
\end{itemize}
  
\begin{columns}[T]
  \begin{column}{0.45\textwidth}
The reconstructed particles are (cuts and selections in the next slide):
    \begin{enumerate}[(a)]
      \item $p + \pi^- + (\gamma_{ISR}) $ which compose the recoil system 
      \item Neutral clusters associated to $\bar{n}$ candidates list used to compare its variables with those of the recoil
    \end{enumerate}  
\end{column}

  \begin{column}{0.50\textwidth}
    \begin{figure}[p]
         \includegraphics[width = 0.95 \textwidth]{images/topoana_100k}
    \end{figure}
  \end{column}

\end{columns}
\vspace{0.3cm}
\begin{itemize}
\item \textbf{100k events}. The reconstruction efficiency is:
  \vspace{0.1cm}
\begin{center}
  $\epsilon = \frac{n° \ of \ reconstructed \ candidates}{n° \ of \ generated \ events} \sim 22\% (18\% )$
\end{center}


\end{itemize}
\end{frame}

%------------------------------------------------------------------------------------------------
\begin{frame}{Applied selections and cuts}


\begin{columns}[T] % T = top alignment
    	\begin{column}{0.55\textwidth}
	\begin{enumerate}[(a)]
	 	
		{
			\item \textbf{proton}: standard PID selection, with tracks required to originate from the IP
			\vspace{0.1cm}
			\item \textbf{pion}: standard PID selection
			\vspace{0.1cm}
			\item \textbf{anti-neutron}: neutral clusters from ECL
			\vspace{0.1cm}
			\item  0 GeV < recoil mass < 2 GeV
			\vspace{0.1cm}
			\item $\alpha$ < 0.35 rad ($\sim 20$ deg)
		}
	
	\end{enumerate}
	\end{column}

	\begin{column}{0.55\textwidth}
		\begin{figure}[p]
     		   	\includegraphics[width = 0.95\textwidth]{images/signal_sample/gen_alpha.pdf}
		\end{figure}
	\end{column}
\end{columns}

Where $\alpha$ is the angle between the recoil vector direction and the closest $\bar{n}$ cluster


\end{frame}

%------------------------------------------------------------------------------------------------
\begin{frame}{The recoil mass}

\begin{columns}[T] % T = top alignment
    % Colonna sinistra
    	\begin{column}{0.50\textwidth}
	 	\begin{itemize}
		{
			\item The recoil mass is well reconstructed in both ISR ($p, \pi^-, \gamma_{ISR}$) and no ISR  ($p, \pi^-$) cases
			\vspace{0.3cm}
			\\ $\rightarrow$ Variables associated with the $\bar{n}$ candidate clusters can be compared with the reconstructed recoil variables ($p$, $\theta$)
			\vspace{0.3cm}
			\item Since there is more than one $\gamma_{ISR}$ per event, the \inlineimg{images/logos/blue} distribution shows a higher number of entries in the left tail

		}
		\end{itemize}
	\end{column}

	\begin{column}{0.55\textwidth}
		\begin{figure}[p]
		   	\includegraphics[width = 0.95\textwidth]{images/signal_sample/gen_mRecoil.pdf}
		\end{figure}
	\end{column}
\end{columns}

\end{frame}


%------------------------------------------------------------------------------------------------

\begin{frame}{The recoil and the $\bar{n}$ momentum}
			
		\begin{figure}[p]
		   	\includegraphics[width = 0.45\textwidth]{images/signal_sample/pRecoil_nmcP_isr.pdf}
			\includegraphics[width = 0.45\textwidth]{images/signal_sample/pRecoil_nmcP_nog.pdf}
		\end{figure}		
	 	\begin{itemize}
		{
			\item The $\bar{n}$ candidate list shows a discrepancy with the recoil momentum
			\begin{enumerate}[(a)]
			{
			\item Could \textcolor{red}{cluster split-off}  and/or \textcolor{red}{pre-showering} introduce the left peak?
			\item \textcolor{red}{Do they affect MC association too?}
			}
			\end{enumerate}
		}		
		\end{itemize}
		



\end{frame}

%------------------------------------------------------------------------------------------------

\begin{frame}{The recoil and the $\bar{n}$ momentum}
 	\begin{itemize}
		{
			\item MC truth can be imposed by selecting only correctly reconstructed $\bar{n}$ candidates from the $\bar{n}$ list

			\item A good correspondence can be observed between the two distributions, in both the ISR and NO ISR cases

		}		
		\end{itemize}


		\begin{figure}[p]
		   	\includegraphics[width = 0.45\textwidth]{images/signal_sample/pRecoil_nmcP_PDGsel_isr.pdf}
			\includegraphics[width = 0.45\textwidth]{images/signal_sample/pRecoil_nmcP_PDGsel_nog.pdf}
		\end{figure}	
\end{frame}
%-------------------------------------------------------------------------------------------------------------------------------------------------------------------------
\begin{frame}{$\bar{n}$ vs recoil vector correlation}

		\begin{figure}[p]
		  	\includegraphics[width = 0.24\textwidth]{images/signal_sample/gen_mc_theta_corr_isr.pdf}
			\includegraphics[width = 0.24\textwidth]{images/signal_sample/gen_rec_theta_corr_isr.pdf}
			\includegraphics[width = 0.24\textwidth]{images/signal_sample/gen_mc_p_corr_isr.pdf}
			\includegraphics[width = 0.24\textwidth]{images/signal_sample/gen_rec_p_corr_isr.pdf}
			
			\includegraphics[width = 0.24\textwidth]{images/signal_sample/gen_mc_theta_corr_nog.pdf}
			\includegraphics[width = 0.24\textwidth]{images/signal_sample/gen_rec_theta_corr_nog.pdf}
			\includegraphics[width = 0.24\textwidth]{images/signal_sample/gen_mc_p_corr_nog.pdf}
			\includegraphics[width = 0.24\textwidth]{images/signal_sample/gen_rec_p_corr_nog.pdf}
		\end{figure}
		
\begin{itemize}		
\item Poor resolution is observed in reconstructed $\bar{n}$ momentum ($p_{rec}^2 = E_{ECL}^2 - m_{\bar{n}}^2$), in both the ISR and NO ISR cases 
\end{itemize}
\end{frame}

%-------------------------------------------------------------------------------------------------------------------------------------------------------------------------
\begin{frame}{$\bar{n}$ vs recoil vector residuals}
	\begin{columns}[T] % T = top alignment
    % Colonna sinistra
    	\begin{column}{0.30\textwidth}
	 	\begin{itemize}
		{
			\item Good correlation is observed at the generator level in both the momentum and $\theta$ distributions
			\vspace{0.2cm}
			\item Besides exhibiting poor resolution, the reconstructed momentum of the $\bar{n}$ is underestimated $\rightarrow$ \textcolor{red}{missing energy in the shower} 
		}
		\end{itemize}
	\end{column}

	\begin{column}{0.65\textwidth}
		\begin{figure}[p]
			\includegraphics[scale=0.22, angle=0]{images/signal_sample/diff_thetaREC_ntheta_mc.pdf}
		  	\includegraphics[scale=0.22, angle=0]{images/signal_sample/diff_thetaREC_ntheta_rec.pdf}
			\includegraphics[scale=0.22, angle=0]{images/signal_sample/diff_pREC_np_mc.pdf}
			\includegraphics[scale=0.22, angle=0]{images/signal_sample/diff_pREC_np_rec.pdf}
			
		\end{figure}
	\end{column}
	\end{columns}
		
\end{frame}

%-------------------------------------------------------------------------------------------------------------------------------------------------------------------------
\begin{frame}{KInematic Fit over the recoil mass}
A 1C kinematic fit can possibly be used to add a constraint and improve the recoil resolution:

	 	\begin{itemize}
		{
			\item No significant differences can be seen in $\theta_{recoil}$ vs MC $\theta_{\bar{n}}$
			\item An improvement can be observed in in $p_{recoil}$ vs MC $p_{\bar{n}}$
		}
		\end{itemize}
		\begin{figure}
			\includegraphics[width = 0.45\textwidth]{images/kin/kin_comp_deltaTheta.pdf}
		  	\includegraphics[width = 0.45\textwidth]{images/kin/kin_comp_deltaP.pdf}
		\end{figure}
		
\end{frame}

%-----------------------------------------------------------------------------------------------------------------------------------------------------------------
\begin{frame}{$\bar{n}$ ECL cluster variables}
The aim is to validate several $\bar{n}$ shower shapes variables, such as:

\begin{itemize}
\begin{columns}  
  \begin{column}{0.30\textwidth}
     {
    \item \textbf{Energy} of the cluster
    \item \textbf{E1E9} and \textbf{E9E21} 
    \item \textbf{Zernike Moments} 
     }
   \end{column}
   
   \begin{column}{0.55\textwidth}
  		\begin{figure}
		  	\includegraphics[width = 0.45\textwidth]{images/cluster/zernike.png}
			\includegraphics[width = 0.45\textwidth]{images/cluster/cluster}
		\end{figure}
    \end{column}
    \end{columns}
   
   \vspace{0.2cm}
   
		\item \textbf{Lateral momentum} and \textbf{Second moment} defined as:
		\begin{center}
		$C_{LM} = \frac{ \sum\limits_{i=2}^n \omega_i E_i r_i^2} { \omega_0 E_0 r_0^2 + \omega_1 E_1 r_1^2+ \sum\limits_{i=2}^n \omega_i E_i r_i^2} \hspace{2cm} C_{SM} = \frac{ \sum\limits_{i=0}^n \omega_i E_i r_i^2} { \sum\limits_{i=0}^n \omega_i E_i}$  
    		\end{center}
    
\end{itemize}

\end{frame}

%-----------------------------------------------------------------------------------------------------------------------------------------------------------------
\begin{frame}{$\bar{n}$ ECL cluster variables}
\inlineimg{images/logos/blue} all $\bar{n}$ candidates \hspace{1.5 cm} \inlineimg{images/logos/red} $\bar{n}$ MC truth ID 
 \begin{figure}
	\includegraphics[width=0.31\textwidth]{images/cluster/clusterE}
	\includegraphics[width=0.31\textwidth]{images/cluster/clusterE1E9}
	\includegraphics[width=0.31\textwidth]{images/cluster/clusterE9E21}
	
	\includegraphics[width=0.31\textwidth]{images/cluster/clusterAbsZernikeMoment51}
	\includegraphics[width=0.31\textwidth]{images/cluster/clusterLAT}
	\includegraphics[width=0.31\textwidth]{images/cluster/clusterSecondMoment}
\end{figure}

\end{frame}

%-----------------------------------------------------------------------------------------------------------------------------------------------------------------
\begin{frame}{$\bar{n}$ ECL cluster variables }

\begin{columns}  
  \begin{column}{0.35\textwidth}
     {
     \begin{enumerate}
     \item Could secondary clusters (split-off) be mis-identified as $\bar{n}$ primary cluster, during reconstruction?
     \item Could it be a wrong association due to pre-showering in the TOP detector?
     \end{enumerate}
     }
     $\rightarrow$ \textcolor{red}{further studies can be addressed with a particle gun generator}
   \end{column}
   
   \begin{column}{0.60\textwidth}
   $\bar{n}$ clusters mainly involve 15 or more crystals
 \begin{figure}
	\includegraphics[width=1\textwidth]{images/cluster/clusterNHits}
\end{figure}
    \end{column}
    \end{columns}

\end{frame}

%-----------------------------------------------------------------------------------------------------------------------------------------------------------------
\begin{frame}{Summary (1)}
\begin{itemize}
\item Channel  $e^+ + e^-  \rightarrow p + \bar{n} + \pi^- + (\gamma_{ISR})$ has been studied and recoil system is correctly reconstructed from the secondary background 

\vspace{0.2cm}

\item Reconstructed $\bar{n}$ variables, such as momentum and shower-shape variables, should be further investigated. This could be due to:

\begin{enumerate}[(a)]
\item \textcolor{red}{Cluster split-offs with the ECL and pre-showering occurring within the TOP may affect the shower shape variables}
\item \textcolor{red}{An incorrect Monte Carlo association could also influence these variables}

\vspace{0.2cm}

$\rightarrow$ \textcolor{red}{This could be studied by testing the ECL response with a $\bar{n}$ particle gun generator}
\end{enumerate}

\vspace{0.2cm}

\item 1C kinematic fit can be possibly adopted during MC/Data comparison, in order to improve the recoil resolution

\end{itemize}
\end{frame}


%------------------------------------------------------------------------------------------------
\makesection{Study of Monte Carlo cocktail}
%------------------------------------------------------------------------------------------------

\begin{frame}{Analysis outline (2)}
\begin{itemize}
\item Study of cocktail  using the following MC sample:

$q \bar{q}$ cocktail with $341 \ M$ events (Luminosity: $215 \ fb^{-1}$)


\vspace{0.2cm}
\item Only the $p$ and $\pi^-$ are used to build the recoil vector (NO ISR for the moment) $\rightarrow$ \textcolor{red}{Can the signal be correctly reconstructed among the other cocktail channels?}
\vspace{0.1cm}
\item Following selections have been applied:

\begin{enumerate}
\item Real particles ID selection on $p$ and $\pi^-$ from the MC truth
\item 0 GeV < recoil mass < 2 GeV
\item The best candidate is selected via $\alpha$ (closest candidate)
\end{enumerate}

\item Same strategy as before:
\begin{enumerate}[(a)]
\item Identify the signal peak near the $\bar{n}$ mass ($\sim 0.939$ GeV) adding further cuts
\item Study the shower shape variables in a selected recoil mass region, replacing the MC truth selections with data driven ones
\item Compare it with data (Data/MC agreement)
\end{enumerate}

\end{itemize}
\end{frame}

%-----------------------------------------------------------------------------------------------------------------------------------------------------------------
\begin{frame}{Recoil mass distribution - MC selections}

\begin{columns}  
  \begin{column}{0.50\textwidth}
     {
     \begin{itemize}
     \item The following cuts are applied to enhance the signal:
     \vspace{0.2cm}
     	\begin{enumerate}[(a)]
	\item No additional charged tracks in the event \inlineimg{images/logos/red} 
	\item No additional charged tracks in the event and alpha<0.35 rad (angular cut on the closest candidate) \inlineimg{images/logos/green} 
	\end{enumerate}
	\vspace{0.4cm}
	\item A peak can be observed above the $\bar{n}$ mass $\rightarrow$ \textcolor{red}{the signal is correctly selected in the MC cocktail }
     \end{itemize}
     }
   \end{column}
   
   
   \begin{column}{0.63\textwidth}
 \begin{figure}
	\includegraphics[width=1\textwidth]{images/continuum/mRecoil_comparison.pdf}
\end{figure}
    \end{column}
    \end{columns}

\end{frame}

%-----------------------------------------------------------------------------------------------------------------------------------------------------------------
\begin{frame}{Recoil mass distribution - Real cuts}

\begin{columns}  
  \begin{column}{0.50\textwidth}
     {
     \begin{itemize}
     \item "Real cuts" can be applied instead of MC truth selections, such as:
     \vspace{0.2cm}
     	\begin{enumerate}[(a)]
	 	
		{
			\item \textbf{proton}: standard PID selection, with tracks required to originate from the IP
			\vspace{0.1cm}
			\item \textbf{pion}: standard PID selection
			\vspace{0.1cm}
		}
	\end{enumerate}
	
	\vspace{0.2cm}
	\item To select the signal, a recoil mass cut in the range (0.8-1.1) GeV can be applied to study the shower shapes variables

     \end{itemize}
     }
   \end{column}
   
   
   \begin{column}{0.63\textwidth}
 \begin{figure}
	\includegraphics[width=1\textwidth]{images/continuum/mRecoil_comparison_real_new.pdf}
\end{figure}
    \end{column}
    \end{columns}

\end{frame}


%-----------------------------------------------------------------------------------------------------------------------------------------------------------------
\begin{frame}{$\bar{n}$ ECL cluster variables}
 \begin{figure}
	\includegraphics[width=0.31\textwidth]{images/cluster/nbar_clusterE_real}
	\includegraphics[width=0.31\textwidth]{images/cluster/nbar_clusterE1E9_real}
	\includegraphics[width=0.31\textwidth]{images/cluster/nbar_clusterE9E21_real}
	
	\includegraphics[width=0.31\textwidth]{images/cluster/nbar_clusterAbsZernikeMoment51_real}
	\includegraphics[width=0.31\textwidth]{images/cluster/nbar_clusterLAT_real}
	\includegraphics[width=0.31\textwidth]{images/cluster/nbar_clusterSecondMoment_real}
\end{figure}

\end{frame}


%-----------------------------------------------------------------------------------------------------------------------------------------------------------------
\begin{frame}{Summary (2)}
\begin{itemize}
\item Study of $q \bar{q}$ cocktail has been performed
\item The recoil mass is correctly reconstructed using both Monte Carlo based selections and data-driven cuts, in the presence of other cocktail channels
\item A further selection on the recoil mass has been applied in order to study the shower shape variables and then validate it in a Data/MC agreement
\end{itemize}
Analysis of a $\Lambda \rightarrow p + \pi^- \ (\bar{\Lambda} \rightarrow \bar{p} + \pi^+)$ sample shows that:
	\begin{figure}[p]
     	   \includegraphics[scale=0.15, angle=0]{images/shanette.png}
	\end{figure}
Poor Data/MC agreement in $\bar{p} \rightarrow$ \textcolor{red}{will it be the same for $\bar{n}$?}
\end{frame}


%------------------------------------------------------------------------------------------------
\makesection{Outlook}
%------------------------------------------------------------------------------------------------
%-----------------------------------------------------------------------------------------------------------------------------------------------------------------
\begin{frame}{Outlook}
	
The following topics will be faced from today:
\begin{enumerate}
\item Study of the Monte Carlo software association and of cluster split-off effects in the ECL using a particle gun
\vspace{0.2cm}
\item Study the entire cocktail production, adding further collections
\vspace{0.2cm}
\item Study of the Data/MC agreement of the shower-shape variables using the evaluated real selections
\end{enumerate}



\end{frame} 
%-----------------------------------------------------------------------------------------------------------------------------------------------------------------
%-----------------------------------------------------------------------------------------------------------------------------------------------------------------
    

    
    
\finalpagetext{Thank you for your attention}
%----------------------------------------------------------------------------------------
\makefinalpage
%----------------------------------------------------------------------------------------




\backupbegin

%----------------------------------------------------------------------------------------
%----------------------------------------------------------------------------------------

\begin{frame}[allowframebreaks,noframenumbering, plain]{$\bar{n}$ mcPDG }
$\gamma$'s are mis-identified as $\bar{n}$ in reconstruction:
	\begin{figure}
     	   \includegraphics[scale=0.40, angle=0]{images/backup/nbar_mcPDG.pdf}
	\end{figure}	
\end{frame}
%----------------------------------------------------------------------------------------
\begin{frame}[allowframebreaks,noframenumbering, plain]{$p_{\bar{n}}$/pRecoil }
$\bar{n}$ is underrated in the most of cases (annihilation process + loss of energy)
	\begin{figure}
     	   \includegraphics[scale=0.40, angle=0]{images/backup/frac_pREC_np.pdf}
	\end{figure}	
\end{frame}
%----------------------------------------------------------------------------------------
\begin{frame}[allowframebreaks,noframenumbering, plain]{Recommended PID selections }
PID probability defined as:
	\begin{figure}
     	   \includegraphics[scale=0.40, angle=0]{images/backup/protonID}	
     	   \includegraphics[scale=0.40, angle=0]{images/backup/pionID}
	\end{figure}	
\end{frame}
%----------------------------------------------------------------------------------------
\begin{frame}[allowframebreaks,noframenumbering, plain]{Zernike Moments}

\vspace{0.2cm}
\begin{columns}
\begin{column}{0.4\textwidth}
\begin{itemize}
\item Zernike polynomials are widely used as basis functions of image moments
	\begin{figure}	
		\includegraphics[width = 0.75\textwidth]{images/cluster/zernike.png}
	\end{figure}
\end{itemize}
\end{column}

\begin{column}{0.6\textwidth}
	\begin{figure}	
		\includegraphics[width = 0.75\textwidth]{images/backup/zernike_wiki.png}
	\end{figure}	
\end{column}
\end{columns}
\end{frame}

\backupend

\end{document}