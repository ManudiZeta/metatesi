%----------------------------------------------------------------------------------------
%	PACKAGES AND THEMES
%----------------------------------------------------------------------------------------
\documentclass[aspectratio=169,xcolor=dvipsnames, t]{beamer}
\usepackage{fontspec} % Allows using custom font. MUST be before loading the theme!
\usetheme{SimplePlusAIC}
\usepackage{hyperref}
\usepackage{mathtools}
\usepackage{enumerate}
\usepackage{graphicx} % Allows including images
\usepackage{booktabs} % Allows the use of \toprule, \midrule and  \bottomrule in tables
\usepackage{svg} %allows using svg figures
\usepackage{tikz}
\usepackage{makecell}
% ADD YOUR PACKAGES BELOW
\usepackage{wrapfig}
\usepackage[export]{adjustbox}

\usepackage[
  backend=biber,
  style=numeric,
  sorting=none
]{biblatex}

\addbibresource{references.bib}


\newcommand{\backupbegin}{
   \newcounter{finalframe}
   \setcounter{finalframe}{\value{framenumber}}
}
\newcommand{\backupend}{
   \setcounter{framenumber}{\value{finalframe}}
}

\newcommand{\inlineimg}[1]{%
  \raisebox{-0.2em}{\includegraphics[height=1em]{#1}}%
}
%----------------------------------------------------------------------------------------
%	TITLE PAGE CONFIGURATION
%----------------------------------------------------------------------------------------

\title[short title]{Characterization of Hadronic Showers in the Belle II Electromagnetic Calorimeter}  

\author{Emanuele Zanusso}
\institute[Characterization of Hadronic Showers in the Belle II Electromagnetic Calorimeter - Emanuele Zanusso]{Dipartimento di Fisica
\newline
Università degli Studi di Torino
}
% Your institution as it will appear on the bottom of every slide, maybe shorthand to save space


\date{January 14, 2026} % Date, can be changed to a custom date
%----------------------------------------------------------------------------------------
%	PRESENTATION SLIDES
%----------------------------------------------------------------------------------------

\begin{document}

\maketitlepage

\begin{frame}[t,noframenumbering]{Outline}
    \tableofcontents
\end{frame}

\makesection{Anti-neutrons in physics experiments}

%------------------------------------------------
\begin{frame}{Anti-neutron in HEP experiments}
The $\bar{n}$ plays a key role in several physics measurements, such as: 

    \begin{itemize}    
    \item The neutron e.m. form factor studies in $e^+ + e^- \rightarrow n + \bar{n}$ process
        	\begin{figure}[p]
     	   \includegraphics[scale=0.18, angle=0]{images/feynman-eN.png}
	\end{figure}
  
    \item Some decay channels studied at B-factories which involve $\bar{n}$
    
    \begin{enumerate}
    	\item The hyperons decay channel:
	\begin{center}
		$\bar{\Lambda}^0 \rightarrow \pi^0 + \bar{n}$, \hspace{1cm} $\bar{\Sigma}^- \rightarrow \pi^- + \bar{n}$, \hspace{1cm} $\bar{\Lambda}_c \rightarrow  K_s^0 + \pi^0 +\bar{n} $
	\end{center}
	\end{enumerate}
	
\item Discrimination between other neutral particles ($\gamma$) and $\bar{n}$

    \end{itemize}
\end{frame}

%------------------------------------------------
\begin{frame}{Anti-neutrons in astrophysics}
The $\bar{n}$ also plays a key role in several astrophysics measurements, such as:
\vspace{0.2cm}
    \begin{itemize}    
    \item Studying $\bar{n}$ - anti-hyperon potential to improve the understanding of the equation of state of the neutron stars 
    \vspace{0.1cm}
    \item Investigating dark matter through anti-deuterons ($\bar{D}$) in cosmic rays, produced by dark matter annihilation or decay 
    \vspace{0.3cm}

    $ A_{d.m.} \ + \ B_{d.m.} \ \rightarrow hadrons$ ($n$, $\bar{n}$, $p$, $\bar{p}$ etc...  )
    \\
     $ X_{d.m.} \ \rightarrow$ hadrons ($n$, $\bar{n}$, $p$, $\bar{p}$ etc...  )

    \vspace{0.2cm}
    

   $\bar{D}$ is mainly produced through a coalescence mechanism $\bar{n} + \bar{p} \rightarrow \bar{D}$, 
    \\
    where $\bar{p}$ and $\bar{n}$ are nearby in the phase-space

    \end{itemize}
\end{frame}

%------------------------------------------------
\begin{frame}{The Belle II experiment}
SuperKEKB is an asymmetric $e^+ \ e^-$ collider (Tsukuba, Japan)
\begin{columns}[T] % T = top alignment
    % Colonna sinistra
    \begin{column}{0.53\textwidth}
        \begin{itemize}
            \item 7 GeV electron beam (HER)
            \item 4 GeV positron beam (LER)
            \item Peak Luminosity $\sim 5.1 \times 10^{34} cm^{-2} s^{-1}$ \inlineimg{images/logos/coppa-del-mondo} 
            \item Design Luminosity $\sim 8 \times 10^{35} cm^{-2} s^{-1}$  \\ $\rightarrow$ x40 the Belle's one 
        \end{itemize}
    \end{column}
    
    % Colonna destra
    \begin{column}{0.60\textwidth}
        \includegraphics[scale=0.12, angle=0]{images/superkekb.png}
        \includegraphics[scale=0.24, angle=0]{images/int_lumi.png}
    \end{column}
\end{columns}

\vspace{0.5cm} % Spazio tra le due sezioni

% Parte inferiore con un’unica colonna
It operates mainly around $\Upsilon(4S)$ resonance ($\sim10.58$ GeV):
\begin{itemize}
    \item This resonance decays almost exclusively into entangled couples of $B \ \bar{B} \rightarrow$ B-factory
    \item Several goals: flavour physics, BSM physics, heavy hadrons spectroscopy etc...
\end{itemize}

\end{frame}

%----------------------------------------------------------------------------------------
\begin{frame}{The Belle II experiment}

\includegraphics[width=1\textwidth]{images/belleii_dets}
    
\end{frame}

%----------------------------------------------------------------------------------------

\begin{frame}{The Electromagnetic CaLorimeter}
The ECL plays a central role in this thesis
	\begin{columns}
	\begin{column}{0.45\textwidth}
	\begin{itemize}
		\item Array of \textbf{CsI(Tl)} crystals (8376 $6x6x30 cm^3$ crystals in total)
	       	\item	It covers barrel and end-cap regions ($12°\le\theta\le155°$)
		\item Energy resolution of 4\% @100 MeV and 1.6\% @8 GeV
	\end{itemize}
	\end{column}
	\begin{column}{0.65\textwidth}
	\begin{figure}[p]
\includegraphics[width = 1 \textwidth]{images/belleii_scheme}
	\end{figure}
	\end{column}
	
	\end{columns}
    
\end{frame}
%----------------------------------------------------------------------------------------
\begin{frame}{Anti-neutron interactions in physics}
The $\bar{n}$ interacts primarily via strong nuclear force, producing hadronic showers
\\
It can annihilate with nucleons in the ECL, producing light mesons (mainly pions) 
\vspace{0.2cm}

    \begin{itemize}    
	\item $\pi^0$ decays into $\gamma\gamma$, producing electromagnetic showers that are fully contained in the ECL
	\item $\pi^\pm$ undergo hadronic interactions, which are not fully contained in the ECL $\rightarrow$ both the forward (KLM) and backward (TOP) directions are involved
	\vspace{0.3cm}
    	\begin{figure}[p]
     	   \includegraphics[scale=0.16, angle=0]{images/Ann-nbar.png}
	   \hspace{1.5cm}
	    \includegraphics[scale=0.23, angle=0]{images/shanette_nbar_ECLTOP.jpg}
	\end{figure}
	 
    \end{itemize}
\end{frame}

%------------------------------------------------
\begin{frame}{The MANTRA project (PRIN2022)}
 \textbf{M}easuring \textbf{A}nti-\textbf{N}eutron: \textbf{T}agging and \textbf{R}econstruction \textbf{A}lgorithm
 
 \vspace{0.2cm}
    \begin{itemize}    
    \item A general method to measure the $E_{\bar{n}}$ up to 10 GeV, by combining information from:
    
    	\begin{enumerate}
    		\item A  detector with high time resolution (TOP)
		\item An electromagnetic calorimeter (\textbf{ECL})
		\item A muon system (KLM)
    	\end{enumerate}
	 \vspace{0.1cm}
     \item These features are common in modern general-purpose collider experiments such as \textbf{Belle II} and BESIII, which do not have a dedicated calorimeter 
     \item For MANTRA project, only signals from ECL and TOP are taken into account. In this thesis only ECL signals are studied
    \end{itemize}
    
    
    \vspace{0.4cm}
    {\large
		\textcolor{red}{ \textbf{Are $\bar{n}$ hadronic showers correctly simulated in the Belle II software?}}
    }
\end{frame}

%------------------------------------------------
\begin{frame}{The MANTRA project} LA LEVO?????
Anti-neutrons cannot be reconstructed by sub-detectors.
\\
The measurement of the energy is a two-step process:
\vspace{0.1cm}
    	\begin{enumerate}
    		\item $\bar{n}$ identification via its induced ECL clusters (study of the shower shape)
		\item Combine the signals from TOP and ECL to reconstruct the $\bar{n}$ energy, in cases of backscattering or pre-showering
		\vspace{0.1cm}
		\begin{itemize}
			\item If $\pi^0$ ($\sim 5 \%$): energy is all contained in the calorimeter, the shower is fully reconstructed
		        	\item	If $\pi^{\pm}$ ($\sim 95\%$): their products may escape the crystals \\ $\rightarrow$ the goal is to complement the calorimeter information with that from the adjacent detectors
		\end{itemize}
    	\end{enumerate}
    
\end{frame}

%------------------------------------------------
\begin{frame}{Preliminary concept}
Several channels can be selected to look at $\bar{n}$ annihilations, such as:
	\begin{itemize}
		\item $e^+ + e^-  \rightarrow p + \bar{n} + \pi^- + (\gamma_{ISR})$ (Mine)
		\item $\bar{\Lambda}_c \rightarrow  K_s^0 + \pi^0 +\bar{n} $ 
		\item $\Lambda (\rightarrow \ p + \pi^- ) + \bar{\Lambda} (\rightarrow \bar{n} + \pi^0)$ 
	\end{itemize}
	
	\vspace{0.2cm}
Several variables can be used to validate the showers shape for $\bar{n}$ identification,$\bar{n}$ identification via its induced ECL clusters (study of the shower shape) such as:
\vspace{0.1cm}
    	\begin{itemize}    
	{ \small
		\item \textbf{Zernike Moments}, which describe cluster shape (backup)
		\item \textbf{Lateral momentum} defined as:  $C_{LM} = \frac{ \sum\limits_{i=2}^n \omega_i E_i r_i^2} { \omega_0 E_0 r_0^2 + \omega_1 E_1 r_1^2+ \sum\limits_{i=2}^n \omega_i E_i r_i^2}$    
 		\item \textbf{Second moment} defined as:  $ C_{SM} = \frac{ \sum\limits_{i=0}^n \omega_i E_i r_i^2} { \sum\limits_{i=0}^n \omega_i E_i}$  
	}
    	\end{itemize}
	
	
\end{frame}

%------------------------------------------------------------------------------------------------
\makesection{Preliminary study via signal Monte Carlo sample}
%------------------------------------------------------------------------------------------------

\begin{frame}{Analysis outline}
\begin{enumerate}
	\item Study of the selected signal channel $e^+ + e^- \rightarrow p + \bar{n} + \pi^- + (\gamma_{ISR})$
    		\begin{enumerate}[(a)]
    			\item Recoil identification from the system $p + \pi^-$ (with and without ISR) 
			\item Study of the kinematic recoil variables (momentum, angles, energy, etc...)
			\item Study of the effect of 1C kinematic fit over the recoil mass
			\item Study of ECL shower shape variables
    		\end{enumerate}
		
	\vspace{0.2cm}
	\item Study of MC cocktail sample:

	\begin{enumerate}[(a)]
			\item Recoil identification from the system $p + \pi^-$ (with and without ISR) 
			\item Study of the kinematic recoil variables (momentum, angles, energy, etc...)
	\end{enumerate}
	
	\vspace{0.2cm}
	\item Study of real data sample:
	\begin{enumerate}[(a)]
			\item Recoil identification from the system $p + \pi^-$ (with and without ISR) 
			\item Constraint with 1C kinematic fit over the recoil mass
			\item Examine Data/MC agreement in ECL cluster shapes from $\bar{n}$ channel

	\end{enumerate}
			
	
\end{enumerate}
\end{frame}

%------------------------------------------------------------------------------------------------

\begin{frame}{Analysis outline (1)}
\begin{itemize}

\item The analyzed channel is:

  \begin{center}
    $e^+ + e^-  \rightarrow p + \bar{n} + \pi^- + (\gamma_{ISR})$ 
  \end{center}
\end{itemize}
  
\begin{columns}[T]
  \begin{column}{0.45\textwidth}
The reconstructed particles are (cuts and selections in the next slide):
    \begin{enumerate}[(a)]
      \item $p + \pi^- + (\gamma_{ISR}) $ which compose the recoil system 
      \item Neutral clusters associated to $\bar{n}$ candidates list used to compare its variables with those of the recoil
    \end{enumerate}  
\end{column}

  \begin{column}{0.50\textwidth}
    \begin{figure}[p]
         \includegraphics[width = 0.95 \textwidth]{images/topoana_100k}
    \end{figure}
  \end{column}

\end{columns}
\vspace{0.3cm}
\begin{itemize}
\item \textbf{100k events}. The reconstruction efficiency is:
  \vspace{0.1cm}
\begin{center}
  $\epsilon = \frac{n° \ of \ reconstructed \ candidates}{n° \ of \ generated \ events} \sim 22\% (18\% )$
\end{center}


\end{itemize}
\end{frame}

%------------------------------------------------------------------------------------------------
\begin{frame}{Applied selections and cuts}


\begin{columns}[T] % T = top alignment
    	\begin{column}{0.55\textwidth}
	\begin{enumerate}[(a)]
	 	
		{
			\item \textbf{proton}: standard PID selection, with tracks required to originate from the IP
			\vspace{0.1cm}
			\item \textbf{pion}: standard PID selection
			\vspace{0.1cm}
			\item \textbf{anti-neutron}: neutral clusters from ECL
			\vspace{0.1cm}
			\item  0 GeV < recoil mass < 2 GeV
			\vspace{0.1cm}
			\item $\alpha$ < 0.35 rad ($\sim 20$ deg)
		}
	
	\end{enumerate}
	\end{column}

	\begin{column}{0.55\textwidth}
		\begin{figure}[p]
     		   	\includegraphics[width = 0.95\textwidth]{images/signal_sample/gen_alpha.pdf}
		\end{figure}
	\end{column}
\end{columns}

Where $\alpha$ is the angle between the recoil vector direction and the closest $\bar{n}$ cluster


\end{frame}

%------------------------------------------------------------------------------------------------
\begin{frame}{The recoil mass}

\begin{columns}[T] % T = top alignment
    % Colonna sinistra
    	\begin{column}{0.50\textwidth}
	 	\begin{itemize}
		{
			\item The recoil mass is well reconstructed in both ISR ($p, \pi^-, \gamma_{ISR}$) and no ISR  ($p, \pi^-$) cases
			\vspace{0.3cm}
			\\ $\rightarrow$ Variables associated with the $\bar{n}$ candidate clusters can be compared with the reconstructed recoil variables ($p$, $\theta$)
			\vspace{0.3cm}
			\item Since there is more than one $\gamma_{ISR}$ per event, the \inlineimg{images/logos/blue} distribution shows a higher number of entries in the left tail

		}
		\end{itemize}
	\end{column}

	\begin{column}{0.55\textwidth}
		\begin{figure}[p]
		   	\includegraphics[width = 0.95\textwidth]{images/signal_sample/gen_mRecoil.pdf}
		\end{figure}
	\end{column}
\end{columns}

\end{frame}


%------------------------------------------------------------------------------------------------

\begin{frame}{The recoil and the $\bar{n}$ momentum}

\begin{columns}[T] % T = top alignment
    % Colonna sinistra
    	\begin{column}{0.40\textwidth}
	 	\begin{itemize}
		{
			\item The reconstructed $\bar{n}$ candidate list shows a discrepancy with the recoil momentum
			$\rightarrow$ several $\gamma$ are mis-identified as $\bar{n}$ in reconstruction 
			\vspace{0.2cm}
			\item MC selection $\bar{n}_{mcPDG} = -2112$ is applied in order to directly compare the recoil kinematic variables with the $\bar{n}$ from MC truth (next slide)

		}
		\end{itemize}
	\end{column}

	\begin{column}{0.55\textwidth}
		\begin{figure}[p]
		   	\includegraphics[scale=0.40, angle=0]{images/pRecoil_nmcP.pdf}
		\end{figure}
	\end{column}
\end{columns}

\end{frame}

%------------------------------------------------------------------------------------------------

\begin{frame}{The recoil and the $\bar{n}$ momentum}

\begin{columns}[T] % T = top alignment
    % Colonna sinistra
    	\begin{column}{0.40\textwidth}
	 	\begin{itemize}
		{
			\item Among the 17525 reconstructed candidates, 6905 correspond to real $\bar{n}$.
			\begin{enumerate}[(a)]
				\item 100000 generated events
				\item 17525 reconstructed events ($\sim 18\%$)
				\item 6905 real $\bar{n}$ in candidates list ($\sim 7\%$)
			\end{enumerate}
			
			\item For a $(LUMI)$ real data, $(TOT events)$ are expected
		}
		\end{itemize}
	\end{column}

	\begin{column}{0.55\textwidth}
		\begin{figure}[p]
		   	\includegraphics[scale=0.40, angle=0]{images/pRecoil_nmcP_PDGsel.pdf}
		\end{figure}
	\end{column}
\end{columns}

\end{frame}
%-------------------------------------------------------------------------------------------------------------------------------------------------------------------------
\begin{frame}{$\bar{n}$ vs recoil vector correlation}
	\begin{columns}[T] % T = top alignment
    % Colonna sinistra
    	\begin{column}{0.30\textwidth}
	 	\begin{itemize}
		{
			\item Good correlation is observed at the generator level in both the momentum and $\theta$ distributions
			\vspace{0.2cm}
			\item The reconstructed $\bar{n}$ momentum in the ECL is not a reliable variable, since no high correlation is observed (annihilation and energy loss)
		}
		\end{itemize}
	\end{column}

	\begin{column}{0.65\textwidth}
		\begin{figure}[p]
		  	\includegraphics[scale=0.22, angle=0]{images/gen_mc_theta_corr.pdf}
			\includegraphics[scale=0.22, angle=0]{images/gen_rec_theta_corr.pdf}
			\includegraphics[scale=0.22, angle=0]{images/gen_mc_p_corr.pdf}
			\includegraphics[scale=0.22, angle=0]{images/gen_rec_p_corr.pdf}
		\end{figure}
	\end{column}
	\end{columns}
		
\end{frame}

%-------------------------------------------------------------------------------------------------------------------------------------------------------------------------
\begin{frame}{$\bar{n}$ vs recoil vector residuals}
	\begin{columns}[T] % T = top alignment
    % Colonna sinistra
    	\begin{column}{0.30\textwidth}
	 	\begin{itemize}
		{
			\item Good correlation is observed at the generator level in both the momentum and $\theta$ distributions
			\vspace{0.2cm}
			\item The reconstructed $\bar{n}$ momentum in the ECL is not a reliable variable, since no high correlation is observed (annihilation and energy loss)
		}
		\end{itemize}
	\end{column}

	\begin{column}{0.65\textwidth}
		\begin{figure}[p]
			\includegraphics[scale=0.22, angle=0]{images/diff_thetaREC_ntheta_mc.pdf}
		  	\includegraphics[scale=0.22, angle=0]{images/diff_thetaREC_ntheta_rec.pdf}
			\includegraphics[scale=0.22, angle=0]{images/diff_pREC_np_mc.pdf}
			\includegraphics[scale=0.22, angle=0]{images/diff_pREC_np_rec.pdf}
			
		\end{figure}
	\end{column}
	\end{columns}
		
\end{frame}

%-------------------------------------------------------------------------------------------------------------------------------------------------------------------------
\begin{frame}{KInematic Fit over the recoil mass}
A 1C kinematic fit can possibly be used to add a constraint and improve the agreement in $p$ and $\theta$

	 	\begin{itemize}
		{
			\item Highest amount of reconstructed candidates ($\sim 24 \%$) and of real $\bar{n}$ ($\sim 9 \%$)
			\item No significant differences can be seen in $\theta_{recoil}$ vs MC $\theta_{\bar{n}}$
			\item An improvement can be observed in in $p_{recoil}$ vs MC $p_{\bar{n}}$
		}
		\end{itemize}
		\begin{figure}
			\includegraphics[scale=0.32, angle=0]{images/kin/kin_comp_deltaTheta.pdf}
		  	\includegraphics[scale=0.32, angle=0]{images/kin/kin_comp_deltaP.pdf}
		\end{figure}
		
\end{frame}

%-----------------------------------------------------------------------------------------------------------------------------------------------------------------
\begin{frame}{$\bar{n}$ ECL cluster variables}
Shower shapes variables can be studied to distinguish $\bar{n}$ from other neutral particles:

\begin{itemize}
\begin{columns}  
  \begin{column}{0.30\textwidth}
     {
    \item \textbf{E}, \textbf{E1E9} and \textbf{E9E21} ($E_{min} = 20$ GeV)
     \item \textbf{ZernikeMoment51}: $|Z_{51}|$ 
     }
   \end{column}
   
   \begin{column}{0.70\textwidth}
  		\begin{figure}[p]
		  	\includegraphics[width = 0.40\textwidth]{images/cluster/zernike.png}
			\includegraphics[width = 0.40\textwidth]{images/cluster/cluster}
		\end{figure}
    \end{column}
    \end{columns}
   
    \item \textbf{Lateral momentum}:  lateral energy distribution, defined as:  $S = \frac{ \sum\limits_{i=2}^n \omega_i E_i r_i^2} { \omega_0 E_0 r_0^2 + \omega_1 E_1 r_1^2+ \sum\limits_{i=2}^n \omega_i E_i r_i^2}$    
    \item \textbf{Second moment}: distribution S, defined as:  $ S = \frac{ \sum\limits_{i=0}^n \omega_i E_i r_i^2} { \sum\limits_{i=0}^n \omega_i E_i}$  
    
\end{itemize}

\end{frame}

%-----------------------------------------------------------------------------------------------------------------------------------------------------------------
\begin{frame}{$\bar{n}$ ECL cluster variables}
 \begin{figure}
	\includegraphics[width=0.31\textwidth]{images/cluster/clusterE}
	\includegraphics[width=0.31\textwidth]{images/cluster/clusterE1E9}
	\includegraphics[width=0.31\textwidth]{images/cluster/clusterE9E21}
	
	\includegraphics[width=0.31\textwidth]{images/cluster/clusterAbsZernikeMoment51}
	\includegraphics[width=0.31\textwidth]{images/cluster/clusterLAT}
	\includegraphics[width=0.31\textwidth]{images/cluster/clusterSecondMoment}
\end{figure}

\end{frame}

%-----------------------------------------------------------------------------------------------------------------------------------------------------------------
\begin{frame}{$\bar{n}$ ECL cluster variables }
\begin{columns}  

  \begin{column}{0.35\textwidth}
     {
     \begin{itemize}
     \item $\bar{n}$ clusters mainly involve 15 or more crystals
     \item Several photons are mis-identified as $\bar{n}$ during reconstruction (backup) $\rightarrow$ further selection can be studied such as:
     \begin{center}
     $\bar{n}\_mcPDG \neq 22$ \\ \&\&  \\ $\bar{n}\_clusterNHits > 15$
     \end{center}
     \end{itemize}
     }
   \end{column}
   \begin{column}{0.60\textwidth}
 \begin{figure}
	\includegraphics[width=1\textwidth]{images/cluster/clusterNHits}
\end{figure}
    \end{column}
    \end{columns}

\end{frame}

%-----------------------------------------------------------------------------------------------------------------------------------------------------------------
\begin{frame}{Summary (1)}
\begin{itemize}
\item Channel  $e^+ + e^-  + \gamma_{ISR} \rightarrow X \rightarrow p + \bar{n} + \pi^-$ has been studied
\item The recoil three body system ($p + \pi^- + \gamma_{ISR}$) is correctly reconstructed from the secondary background, ISR/FSR photons
\item The $\bar{n}$ kinematic is properly described by the three body system $p, \pi^-, \gamma$ recoil vector
\item Reconstructed $\bar{n}$ variables are mainly affected by mis-identified photons, which can be partially cleaned by cluster size cuts ($clusterNHits$)
\item 1C kinematic fit can be possibly adopted during MC/Data comparison, in order to improve the recoil on the recoil momentum

\end{itemize}
\end{frame}


%------------------------------------------------------------------------------------------------
\makesection{Study of Monte Carlo cocktail}
%------------------------------------------------------------------------------------------------

\begin{frame}{Analysis outline (2)}
\begin{itemize}
\item Study of cocktail from MC16rd\_proc16 using the following MC sample:
\begin{enumerate}[(a)]
	\item $q \bar{q}$ cocktail powered by Pythia
	\item Number of Events: $341 \ M$ 
	\item Luminosity: $215 \ fb^{-1}$
\end{enumerate}

\vspace{0.2cm}
\item To obtain a first attempt, only the $p$ and $\pi^-$ are used to build the recoil vector (ISR neglected for the moment)
\item 2345 jobs have been submitted to the grid, with the following online cuts:
\begin{enumerate}
\item  $p\_mcPDG == 2212$ and $pi\_mcPDG == -211$ and 0 GeV < mRecoil < 2 GeV
\item The best candidate is selected with RankByLowest method on $\alpha$ (backup)
\end{enumerate}

\item Same strategy as before:
\begin{enumerate}[(a)]
\item Identify the signal peak near the $\bar{n}$ mass ($\sim 0.939$ GeV) adding offline cuts
\item Study the previous variables (recoil and cluster) in a mRecoil zoomed region 
\item (3) Compare it with data (Data/MC agreement)
\end{enumerate}

\end{itemize}
\end{frame}

%-----------------------------------------------------------------------------------------------------------------------------------------------------------------
\begin{frame}{mRecoil distribution}
\begin{columns}  

  \begin{column}{0.50\textwidth}
     {
     \begin{itemize}
     \item The following cuts are applied to enhance the signal:
     \vspace{0.2cm}
     	\begin{enumerate}[(a)]
	\item nRoeCharged == 0 (no additional charged particles in the Rest Of Event) \inlineimg{images/logos/red} 
	\item nRoeCharged == 0 and alpha<0.35 (additional angular cut on the closest candidate) \inlineimg{images/logos/green} 
	\item nRoeCharged == 0 and alpha<0.35 and nbar\_mcPDG == -2112 (MC truth selection) \inlineimg{images/logos/yellow} 
	\end{enumerate}

     \end{itemize}
     }
   \end{column}
   
   
   \begin{column}{0.63\textwidth}
 \begin{figure}
	\includegraphics[width=1\textwidth]{images/continuum/mRecoil_comparison.pdf}
\end{figure}
    \end{column}
    \end{columns}

\end{frame}

%-----------------------------------------------------------------------------------------------------------------------------------------------------------------
\begin{frame}{mRecoil distribution}
\begin{columns}  

  \begin{column}{0.50\textwidth}
     {
     \begin{itemize}
     \item "Real selections" can be applied as well, with (b), such as: \cite{longo2025}
     \vspace{0.2cm}
\\  protonID > 0.9 and pionID > 0.1 and
\\ dr < 1 and abs(dz) < 3 (from IP)  \inlineimg{images/logos/cyan}
	
	\vspace{0.2cm}
	\item To maximize purity, a recoil mass selection in the range (0.8-1.1) GeV can be applied to study the recoil variables
	\item This set of selections will be applied in the following sections

     \end{itemize}
     }
   \end{column}
   
   
   \begin{column}{0.63\textwidth}
 \begin{figure}
	\includegraphics[width=1\textwidth]{images/continuum/mRecoil_comparison_real.pdf}
\end{figure}
    \end{column}
    \end{columns}

\end{frame}


%-----------------------------------------------------------------------------------------------------------------------------------------------------------------
\begin{frame}{Variables}



\end{frame}

%-----------------------------------------------------------------------------------------------------------------------------------------------------------------
\begin{frame}{Summary (2)}



\end{frame}


%-----------------------------------------------------------------------------------------------------------------------------------------------------------------
\begin{frame}{Data/MC agreement $\bar{n}$ case}
Analysis of a $\Lambda \rightarrow p + \pi^- \ (\bar{\Lambda} \rightarrow \bar{p} + \pi^+)$ sample shows that \cite{shanette2025}:
	\begin{figure}[p]
     	   \includegraphics[scale=0.25, angle=0]{images/shanette.png}
	\end{figure}
Poor Data/MC agreement in $\bar{p} \rightarrow$ will it be the same for $\bar{n}$?
	
\end{frame}


%------------------------------------------------------------------------------------------------
\makesection{Outlook}
%------------------------------------------------------------------------------------------------
%-----------------------------------------------------------------------------------------------------------------------------------------------------------------
\begin{frame}{Outlook Until Graduation}



\end{frame} 
%-----------------------------------------------------------------------------------------------------------------------------------------------------------------
%-----------------------------------------------------------------------------------------------------------------------------------------------------------------
    

    
    
\finalpagetext{Thank you for your attention}
%----------------------------------------------------------------------------------------
\makefinalpage
%----------------------------------------------------------------------------------------




\backupbegin

%------------------------------------------------
\begin{frame}{Electromagnetic and hadronic showers}
Different processes occur for e.m. (1) and hadronic (2) showers:
\vspace{0.1cm}
	\begin{enumerate}
		\item  Bremsstrahlung and pair production process ($e^+,e^-,\gamma$) and $\pi^0 \rightarrow \gamma \gamma$
		\item  Strong interactions of hadrons with the material ($p,n,pions,kaons...$)
	\end{enumerate}

\begin{columns}
	\begin{column}{0.45\textwidth}
    		\begin{itemize}    
			\item About the 95\% of the hadronic shower is contained within a cylinder of radius $\lambda_{had}$ ($\sim$ 44.12 cm in CsI(Tl))
			\item About the 90\% of the e.m. shower is contained within a cylinder of radius $R_M$ ($\sim$ 3.6 cm in CsI(Tl))
	    	\end{itemize}
	\end{column}
	
	\begin{column}{0.60\textwidth}
		\begin{figure}[p]
		   \includegraphics[scale=0.45, angle=0]{images/gamma_vs_hadronic.png}
		\end{figure}
	\end{column}
\end{columns}
\end{frame}

%----------------------------------------------------------------------------------------

\begin{frame}[allowframebreaks,noframenumbering, plain]{$\bar{n}$ mcPDG }
$\gamma$'s are mis-identified as $\bar{n}$ in reconstruction:
	\begin{figure}
     	   \includegraphics[scale=0.40, angle=0]{images/nbar_mcPDG.pdf}
	\end{figure}	
\end{frame}

\begin{frame}[allowframebreaks,noframenumbering, plain]{$p_{\bar{n}}$/pRecoil }
$\bar{n}$ is underrated in the most of cases (annihilation process + loss of energy)
	\begin{figure}
     	   \includegraphics[scale=0.40, angle=0]{images/frac_pREC_np.pdf}
	\end{figure}	
\end{frame}


\backupend


%----------------------------------------------------------------------------------------
\begin{frame}[allowframebreaks,noframenumbering,plain]{References}
  \printbibliography
\end{frame}

\end{document}