%----------------------------------------------------------------------------------------
%	PACKAGES AND THEMES
%----------------------------------------------------------------------------------------
\documentclass[aspectratio=169,xcolor=dvipsnames, t]{beamer}
\usepackage{fontspec} % Allows using custom font. MUST be before loading the theme!
\usetheme{SimplePlusAIC}
\usepackage{hyperref}
\usepackage{mathtools}
\usepackage{enumerate}
\usepackage{graphicx} % Allows including images
\usepackage{booktabs} % Allows the use of \toprule, \midrule and  \bottomrule in tables
\usepackage{svg} %allows using svg figures
\usepackage{tikz}
\usepackage{makecell}
% ADD YOUR PACKAGES BELOW
\usepackage{wrapfig}
\usepackage[export]{adjustbox}

\usepackage[
  backend=biber,
  style=numeric,
  sorting=none
]{biblatex}

\addbibresource{references.bib}


\newcommand{\backupbegin}{
   \newcounter{finalframe}
   \setcounter{finalframe}{\value{framenumber}}
}
\newcommand{\backupend}{
   \setcounter{framenumber}{\value{finalframe}}
}

\newcommand{\inlineimg}[1]{%
  \raisebox{-0.2em}{\includegraphics[height=1em]{#1}}%
}
%----------------------------------------------------------------------------------------
%	TITLE PAGE CONFIGURATION
%----------------------------------------------------------------------------------------

\title[short title]{Characterization of hadronic showers in the Belle II Electromagnetic Calorimeter}  % The short title appears at the bottom of every slide, the full title is only on the title page
\subtitle{Metatesi exam}

\author{Emanuele Zanusso}
\institute[Dipartimento di Fisica di Torino]{Dipartimento di Fisica
\newline
Università degli studi di Torino
}
% Your institution as it will appear on the bottom of every slide, maybe shorthand to save space


\date{January 14 2026} % Date, can be changed to a custom date
%----------------------------------------------------------------------------------------
%	PRESENTATION SLIDES
%----------------------------------------------------------------------------------------

\begin{document}

\maketitlepage

\begin{frame}[t,noframenumbering]{Talk Outline}
    \tableofcontents
\end{frame}

\makesection{Anti-neutron in physics experiment}

%------------------------------------------------
\begin{frame}{Anti-neutron in HEP experiments}
The $\bar{n}$ plays a key role in several physics measurements, such as: 

    \begin{itemize}    
    \item The neutron e.m. form factor studies in $e^+ + e^- \rightarrow n + \bar{n}$ process \cite{ablikim2021}
        	\begin{figure}[p]
     	   \includegraphics[scale=0.18, angle=0]{images/feynman-eN.png}
	\end{figure}
  
    \item The decay channels studied at B-factories which involved $\bar{n}$, such as:
    
    \begin{enumerate}
    	\item The hyperons decay channel:
	\begin{center}
		$\bar{\Lambda}^0 \rightarrow \pi^0 + \bar{n}$, \hspace{1cm} $\bar{\Sigma}^- \rightarrow \pi^- + \bar{n}$, \hspace{1cm} $\bar{\Lambda}_c \rightarrow  K_s^0 + \pi^0 +\bar{n} $
	\end{center}
	
    \item Other typical B-factories processes:
    	\begin{center}
		$e^+ + e^- \rightarrow p + \bar{n} + X^-$ \ \ ($X^-$: combination of charged pions and kaons)
	\end{center}
\end{enumerate}
    \end{itemize}
\end{frame}

%------------------------------------------------
\begin{frame}{Anti-neutron in astrophysical experiments}
$\bar{n}$ also plays a key role in several astro-physics measurements, such as:

    \begin{itemize}    
    \item Studying $\bar{n}$ - anti-hyperon potential to improve the understanding of the equation of state of the neutron star \cite{burgio2021}
    
    \item Investigating dark matter through anti-deuterons ($\bar{D}$) in cosmic rays, produced by dark matter annihilation or decay \cite{donato2000}:
    \\
    $ A_{d.m.} \ + \ B_{d.m.} \ \rightarrow hadrons$ ($n$, $\bar{n}$, $p$, $\bar{p}$ etc...  )
    \\
     $ X_{d.m.} \ \rightarrow$ hadrons ($n$, $\bar{n}$, $p$, $\bar{p}$ etc...  )

    
   \item $\bar{D}$ is mainly produced through a coalescence mechanism:
   
     \begin{center}
    $\bar{n} + \bar{p} \rightarrow \bar{D} $
    \end{center}
    
    Where $\bar{p}$ and $\bar{n}$ are nearby in the phase-space
    \end{itemize}
\end{frame}

%------------------------------------------------
\begin{frame}{The Belle II experiment}

\begin{columns}[T] % T = top alignment
    % Colonna sinistra
    \begin{column}{0.55\textwidth}
        SuperKEKB is an asymmetric $e^+ \ e^-$ collider (Tsukuba, Japan)
        \begin{itemize}
            \item 7 GeV electrons beam (HER)
            \item 4 GeV positrons beam (LER)
            \item Peak Lumi $\sim 5.1 \times 10^{34} cm^{-2} s^{-1}$ \inlineimg{images/coppa-del-mondo} 
            \item Design Lumi $\sim 6.5 \times 10^{35} cm^{-2} s^{-1}$  \\ $\rightarrow$ x40 the Belle's one 
        \end{itemize}
    \end{column}
    
    % Colonna destra
    \begin{column}{0.50\textwidth}
        \includegraphics[scale=0.10, angle=0]{images/superkekb.png}
        \includegraphics[scale=0.20, angle=0]{images/int_lumi.png}
    \end{column}
\end{columns}

\vspace{0.5cm} % Spazio tra le due sezioni

% Parte inferiore con un’unica colonna
It operates mainly around $\Upsilon(4S)$ resonance ($\sim10.58 GeV$):
\begin{itemize}
    \item It decays almost exclusively into entangled couple of $B \ \bar{B} \rightarrow$ B-factory
    \item Several goals: flavour physics, BSM, B and charm mesons etc...
\end{itemize}

\end{frame}
%----------------------------------------------------------------------------------------

\begin{frame}{The Electromagnetic CaLorimeter}
The ECL plays a central role in this thesis
	
	\begin{itemize}
		\item Array of \textbf{CsI(Tl)} crystals (8376 $6x6x30 cm^3$ crystals in total)
	       	\item	It covers barrel and end-caps regions ($12°\le\theta\le155°$)
		\item Energy resolution of 4\% @100 Mev and 1.6\% @8 GeV
	\end{itemize}
	
\includegraphics[scale=0.20, angle=0]{images/belleii_scheme}
\includegraphics[scale=0.20, angle=0]{images/ECL}
    
\end{frame}
%----------------------------------------------------------------------------------------

\begin{frame}{Anti-neutron interactions in physics}
The $\bar{n}$ interacts with matter primarily via strong nuclear force 

    \begin{itemize}    
    \item It can annihilate with nucleons in the material, producing light mesons (mainly pions) 
    \item Hadronic ($\pi^+$, $\pi^- $) and electromagnetic ($\pi^0$) showers are generated within the ECL
    \item Since annihilation stars are produced, both backward (TOP) and forward (KLM) directions are involved
    
    	\begin{figure}[p]
     	   \includegraphics[scale=0.16, angle=0]{images/Ann-nbar.png}
	\end{figure}
	 
    \end{itemize}
\end{frame}
%------------------------------------------------
\begin{frame}{Electromagnetic and hadronic showers}
Different processes occur for e.m. (1) and hadronic (2) showers:
	\begin{enumerate}
		\item  Brems. and p.p. process ($e^+,e^-,\gamma$) and $\pi^0 \rightarrow \gamma \gamma$
		\item  Strong interactions of hadrons with the material ($p,n,pions,kaons...$)
	\end{enumerate}

After $\bar{n}$ annihilation, the produced pions can be detected via their showers:

\begin{columns}
	\begin{column}{0.45\textwidth}
    		\begin{itemize}    
			\item About the 95\% of the hadronic shower is contained within a cylinder of radius $\lambda_{had}$ ($\sim$ 44.12 cm in CsI(Tl))
			\item About the 90\% of the e.m. shower is contained within a cylinder of radius $R_M$ ($\sim$ 3.6 cm in CsI(Tl))
	    	\end{itemize}
	\end{column}
	
	\begin{column}{0.60\textwidth}
		\begin{figure}[p]
		   \includegraphics[scale=0.45, angle=0]{images/gamma_vs_hadronic.png}
		\end{figure}
	\end{column}
\end{columns}
\end{frame}

%------------------------------------------------
\begin{frame}{Anti-neutron interactions in physics}
Several channels can be selected to look at $\bar{n}$ annihilation, such as:
	\begin{itemize}
		\item $e^+ + e^-  \rightarrow X \rightarrow p + \bar{n} + \pi^-$ (Mine)
		\item $\bar{\Lambda}_c \rightarrow  K_s^0 + \pi^0 +\bar{n} $ 
		\item $\Lambda (\rightarrow \ p + \pi^- ) + \bar{\Lambda} (\rightarrow \bar{n} + \pi^0)$ 
	\end{itemize}
	
Several variables can be used to distinguish their clusters, such as:
    	\begin{itemize}    
		\item clusterZernikeMoment, clusterSecondMoment, clusterLAT etc...
    	\end{itemize}
	
	\begin{figure}[p]
     	   \includegraphics[scale=0.22, angle=0]{images/sanjeeda_a.png}
	   \includegraphics[scale=0.2225, angle=0]{images/sanjeeda.png}

	\end{figure}
The distributions for ECL variables for $\bar{p}$ and $\bar{n}$ do not in agree \cite{sanjeeda2025}
$\rightarrow \bar{p}$ cannot be used as proxy for $\bar{n}$
	
\end{frame}

%------------------------------------------------
\begin{frame}{The MANTRA project}
\textbf{M}easuring \textbf{A}nti-\textbf{N}eutron: \textbf{T}agging and \textbf{R}econstruction \textbf{A}lgorithm:

    \begin{itemize}    
    \item A general method to measure the $E_{\bar{n}}$ up to 10 GeV, by combining information from:
    
    	\begin{enumerate}
    		\item A  detector with high time resolution ($<100ps$), like a T.O.F. detector (TOP)
		\item An electromagnetic calorimeter (\textbf{ECL})
		\item A muon system (alternating layers of active material and high-Z absorber) (KLM)
    	\end{enumerate}
     \item These features are common in modern general-purpose collider experiments such as \textbf{Belle II} and BESIII
     \item For MANTRA project, only signals from ECL and TOP are taken into account. In this thesis only ECL signals are studied
    \end{itemize}
\end{frame}

%------------------------------------------------
\begin{frame}{The MANTRA project}
Anti-neutrons do not interact with tracking sub-detector. The measurement of the energy is a two-step process:
    	\begin{enumerate}
    		\item $\bar{n}$ identification via its induced ECL clusters and how hey correlate to the initial energy
		\item Combine the signals from (1) and (2) to reconstruct the $\bar{n}$'s energy, in cases of backscatter or pre-annihilation
		
		\begin{itemize}
			\item If $\pi^0$ ($\sim 5 \%$): energy is all contained in the calorimeter, the shower is fully reconstructed
		        	\item	If $\pi^{\pm}$ ($\sim 95\%$): their products may escape the crystals \\ $\rightarrow$ the goal is to complement the calorimeter information with that from the adjacent detectors
		\end{itemize}
    	\end{enumerate}
    
\end{frame}
%------------------------------------------------------------------------------------------------
\makesection{Preliminary study via clean MC sample}
%------------------------------------------------------------------------------------------------

\begin{frame}{Analysis outline}
\begin{enumerate}
	\item Preliminary study of a clean selected channel via generators
    		\begin{enumerate}[(a)]
    			\item Recoil identification from the three-body system $p + \gamma_{ISR} + \pi^-$ 
			\item Study of the kinematic recoil variables (momentum, angles, energy, etc...)
			\item Study of ECL clusters
			\item Study of the effect of 1C kinematic fit over the $p + \gamma_{ISR} + \pi^-$ recoil mass
    		\end{enumerate}
	
	\item Study of MC cocktail events sample:

	\begin{enumerate}[(a)]
			\item Recoil identification from the three-body system $p + \gamma_{ISR} + \pi^-$ 
	\end{enumerate}
	
	\item Study of real data events sample:
	\begin{enumerate}[(a)]
			\item Recoil identification from the three-body system $p + \gamma_{ISR} + \pi^-$ 
			\item Constraint with 1C kinematic fit over the $p + \gamma_{ISR} + \pi^-$ recoil mass
			\item Examine Data/MC agreement in ECL cluster shapes from $\bar{n}$ channel

	\end{enumerate}
			
	
\end{enumerate}
\end{frame}

%------------------------------------------------------------------------------------------------

\begin{frame}{Analysis outline}
\begin{itemize}

\item Several channels can be selected to study $\bar{n}$ at Belle II experiment. The chosen one is:

	\begin{center}
		$e^+ + e^-  + \gamma_{ISR} \rightarrow X \rightarrow p + \bar{n} + \pi^-$ (Phokhara+evt\_gen generator)
	\end{center}
	
\vspace{0.3cm}

\item The reconstructed particles are (cuts and selections in backup)
	\begin{enumerate}[(a)]
			\item $vpho \rightarrow p + \gamma_{ISR} + \pi^-$, where $vpho$ is a fake particle, mimicking the recoil system 
			\item $\bar{n}$ candidate list, used to compare its variables with those of the recoil
	\end{enumerate}	

\item \textbf{100k events} were generated, \textbf{17525} candidates have been reconstructed 
\\ $\rightarrow$ reconstruction efficiency:

\vspace{0.3cm}

\begin{center} \Large
	$\epsilon = \frac{n° \ of \ reconstructed \ candidates}{n° \ of \ generated \ events} \sim 18\% $
\end{center}

\end{itemize}
\end{frame}

%------------------------------------------------------------------------------------------------

\begin{frame}{The recoil mass (1a)}

\begin{columns}[T] % T = top alignment
    % Colonna sinistra
    	\begin{column}{0.40\textwidth}
	 	\begin{itemize}
		{
			\item The recoil three body system is well reconstructed as shown in the recoil mass distribution, where a peak emerges above the $\bar{n}$ mass.
			\vspace{0.2cm}
			\\ $\rightarrow$ reconstructed $\bar{n}$ variables can be compared with the reconstructed recoil variables ($p, \ \theta, \ \phi $)
		}
		\end{itemize}
	\end{column}

	\begin{column}{0.55\textwidth}
		\begin{figure}[p]
		   	\includegraphics[scale=0.40, angle=0]{images/gen_vpho_r_mRecoil.pdf}
		\end{figure}
	\end{column}
\end{columns}

\end{frame}


%------------------------------------------------------------------------------------------------

\begin{frame}{The recoil and the $\bar{n}$ momentum}

\begin{columns}[T] % T = top alignment
    % Colonna sinistra
    	\begin{column}{0.40\textwidth}
	 	\begin{itemize}
		{
			\item The reconstructed $\bar{n}$ candidate list shows a discrepancy with the recoil momentum
			$\rightarrow \gamma$'s are mis-identified as $\bar{n}$ in reconstruction
			\vspace{0.2cm}
			\item MC selection $\bar{n}_{mcPDG} != 22$ is applied in order to directly compare the recoil kinematic variables with the $\bar{n}$ from MC truth

		}
		\end{itemize}
	\end{column}

	\begin{column}{0.55\textwidth}
		\begin{figure}[p]
		   	\includegraphics[scale=0.40, angle=0]{images/pRecoil_nmcP.pdf}
		\end{figure}
	\end{column}
\end{columns}

\end{frame}

%------------------------------------------------------------------------------------------------

\begin{frame}{The recoil and the $\bar{n}$ momentum}

\begin{columns}[T] % T = top alignment
    % Colonna sinistra
    	\begin{column}{0.40\textwidth}
	 	\begin{itemize}
		{
			\item The reconstructed $\bar{n}$ candidate list shows a discrepancy with the recoil momentum
			$\rightarrow \gamma$'s are mis-identified as $\bar{n}$ in reconstruction
			\vspace{0.2cm}
			\item MC selection $\bar{n}_{mcPDG} != 22$ is applied in order to directly compare the recoil kinematic variables with the $\bar{n}$ from MC truth

		}
		\end{itemize}
	\end{column}

	\begin{column}{0.55\textwidth}
		\begin{figure}[p]
		   	\includegraphics[scale=0.40, angle=0]{images/pRecoil_nmcP_PDGsel.pdf}
		\end{figure}
	\end{column}
\end{columns}

\end{frame}
%------------------------------------------------------------------------------------------------
\begin{frame}{$\bar{n}$ vs recoil vector correlation}
	\begin{columns}[T] % T = top alignment
    % Colonna sinistra
    	\begin{column}{0.30\textwidth}
	 	\begin{itemize}
		{
			\item Good correlation is observed at the generator level in both the momentum and $\theta$ distributions
			\vspace{0.2cm}
			\item The reconstructed $\bar{n}$ momentum in the ECL is not a reliable variable, since no high correlation is observed (energy loss)
		}
		\end{itemize}
	\end{column}

	\begin{column}{0.65\textwidth}
		\begin{figure}[p]
		  	\includegraphics[scale=0.22, angle=0]{images/gen_mc_theta_corr.pdf}
			\includegraphics[scale=0.22, angle=0]{images/gen_rec_theta_corr.pdf}
			\includegraphics[scale=0.22, angle=0]{images/gen_mc_p_corr.pdf}
			\includegraphics[scale=0.22, angle=0]{images/gen_rec_p_corr.pdf}
		\end{figure}
	\end{column}
	\end{columns}
		
\end{frame}



\begin{frame}{Analysis outline}
Analysis of a $\Lambda \rightarrow p + \pi^- \ (\bar{\Lambda} \rightarrow \bar{p} + \pi^+)$ sample shows that \cite{shanette2025}:
	\begin{figure}[p]
     	   \includegraphics[scale=0.25, angle=0]{images/shanette.png}
	\end{figure}
Poor Data/MC agreement in $\bar{p} \rightarrow$ will it be the same for $\bar{n}$?
	
\end{frame}
    
\finalpagetext{Thank you for your attention}
%----------------------------------------------------------------------------------------
\makefinalpage
%----------------------------------------------------------------------------------------




\backupbegin
\begin{frame}[allowframebreaks,noframenumbering, plain]{Applied selections and cuts on clean MC sample}
\vspace{0.6cm}
\begin{columns}[T] % T = top alignment
    % Colonna sinistra
    	\begin{column}{0.50\textwidth}
	 	\begin{enumerate}[(a)]
		{
			\item $protonID > 0.9$ and $dr < 1$ and $abs(dz) < 3$ and $pionID > 0.1$ (From IP?)
			\item  $p\_PDG == 2212$ and $pi\_PDG == -211$ and $gamma\_PDG == 22$
			\item Rec. $\bar{n}$ in theta ECL Acceptance and From ECL
			\item $mRecoil > 0 GeV$ and $mRecoil < 2 GeV$
			\item $\alpha$ < 0.35 rad ($\sim 20$ deg), where $\alpha$ is the 3D angle between the recoil vector and the closest reconstructed $\bar{n}$ candidate (rankByLowest)
		}
		\end{enumerate}
	\end{column}

	\begin{column}{0.55\textwidth}
		\begin{figure}[p]
     		   	\includegraphics[scale=0.35, angle=0]{images/gen_alpha.pdf}
		\end{figure}
	\end{column}
\end{columns}

\end{frame}

\begin{frame}[allowframebreaks,noframenumbering, plain]{$\bar{n}$ mcPDG }
$\gamma$'s are mis-identified as $\bar{n}$ in reconstruction:
	\begin{figure}
     	   \includegraphics[scale=0.40, angle=0]{images/nbar_mcPDG.pdf}
	\end{figure}	
\end{frame}

\begin{frame}[allowframebreaks,noframenumbering, plain]{$p_{\bar{n}}$/pRecoil }
$\bar{n}$ is underrated in the most of cases (annihilation process + loss of energy)
	\begin{figure}
     	   \includegraphics[scale=0.40, angle=0]{images/frac_pREC_np.pdf}
	\end{figure}	
\end{frame}


\backupend

\begin{frame}[allowframebreaks,noframenumbering,plain]{References}
  \printbibliography
\end{frame}

\end{document}